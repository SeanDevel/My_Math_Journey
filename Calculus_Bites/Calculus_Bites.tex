\documentclass[a4paper,12pt]{book}

\usepackage{amsmath}  % For math formulas
\usepackage{amssymb}  % For additional math symbols
\usepackage{amsfonts} % For more math fonts
\usepackage{geometry} % For adjusting page margins
\setcounter{secnumdepth}{2} % Allow numbering to depth 3 (subsubsection)
\geometry{margin=3cm} % 1 inch margins
\usepackage{hyperref} % For hyperlinks within the document
\usepackage{xcolor}
\usepackage{ulem}
\usepackage{tcolorbox}
\tcbuselibrary{skins}

\usepackage{minted}

\definecolor{light_blue}{RGB}{200, 240, 255} % Customize the shade as desired
\definecolor{light_gray}{RGB}{222, 222, 222}

% problem box
\newtcolorbox{bluebox}{
	colback=light_blue,
	colframe=light_blue,
	boxrule=0pt,
	leftrule=2pt,
	rightrule=2pt,
	toprule=2pt,
	bottomrule=2pt,
	sharp corners,
}

\begin{document}
	
\title{Calculus Bites}
\author{Sean}
% \date{\today}

\maketitle

\tableofcontents

\newpage

\chapter{Primary Math}

\section{Inequality}

\begin{bluebox}
Why when \( x\geq 0\) , \( \frac{1}{1+x|\sin{x}|} \geq \frac{1}{1+x} \) ? % [139] p269 t3
\end{bluebox}

To understand why the inequality

\[
\frac{1}{1 + x|\sin x|} \geq \frac{1}{1 + x}
\]

holds for \( x \geq 0 \), let's analyze it step by step:

\textbf{1. Rewrite the inequality}

Rewriting the inequality:

\[
\frac{1}{1 + x|\sin x|} \geq \frac{1}{1 + x}
\]

Multiply through by \( (1 + x|\sin x|)(1 + x) \), which is positive for \( x \geq 0 \), so the direction of the inequality is preserved:

\[
1 + x \geq 1 + x|\sin x|
\]

\textbf{2. Simplify}

Cancel the \( 1 \) on both sides:

\[
x \geq x|\sin x|
\]

Factor \( x \) (since \( x \geq 0 \)):

\[
x(1 - |\sin x|) \geq 0
\]

\textbf{3. Analyze \( 1 - |\sin x| \)}

\begin{itemize}
\item 
The term \( |\sin x| \) satisfies \( 0 \leq |\sin x| \leq 1 \) for all \( x \).

\item 
Hence, \( 1 - |\sin x| \geq 0 \).

\end{itemize}

\textbf{4. Conclusion}

\begin{itemize}
\item 
For \( x \geq 0 \), both \( x \) and \( 1 - |\sin x| \) are non-negative.

\item 
Thus, \( x(1 - |\sin x|) \geq 0 \), confirming the inequality holds.

\end{itemize}

\textbf{Why the inequality is strict}

\begin{itemize}
\item 
If \( x > 0 \) and \( |\sin x| > 0 \), \( 1 + x|\sin x| > 1 + x \), making \( \frac{1}{1 + x|\sin x|} < \frac{1}{1 + x} \).

\end{itemize}


\chapter{Differential}

\section{Derivative}

\begin{bluebox}
Why we can get \(f^{\prime}(0)=0\) from \(\lim_{x\to 0}\frac{f(x)}{x^2}=0\)?
\end{bluebox}

To see why \( f'(0) = 0 \) follows from the given limit, let's carefully analyze the relationship between the limit \(\lim_{x \to 0} \frac{f(x)}{x^2} = 0\) and the definition of the derivative.

\subsubsection{Definitions and reasoning:}
\begin{enumerate}
\item 
\textbf{Limit condition:}\newline
The given condition states:
\[
\lim_{x \to 0} \frac{f(x)}{x^2} = 0.
\]
This implies that \( f(x) \) grows slower than \( x^2 \) as \( x \to 0 \), meaning that \( f(x) \) becomes negligible relative to \( x^2 \) near \( x = 0 \).

\item 
\textbf{Derivative at 0:}\newline
By the definition of the derivative:
\[
f'(0) = \lim_{h \to 0} \frac{f(h)}{h}.
\]
To compute \( f'(0) \), we must examine the behavior of \( \frac{f(h)}{h} \) as \( h \to 0 \).

\item 
\textbf{Relation between \(\frac{f(x)}{x^2}\) and \(\frac{f(x)}{x}\):}\newline
Since \(\lim_{x \to 0} \frac{f(x)}{x^2} = 0\), we can write:
\[
\frac{f(x)}{x^2} \to 0 \quad \text{as } x \to 0.
\]
Multiply both sides by \( x \) (valid for \( x \neq 0 \)) to get:
\[
\frac{f(x)}{x} = x \cdot \frac{f(x)}{x^2}.
\]
Now, as \( x \to 0 \), the factor \( x \) in this expression tends to 0, and \( \frac{f(x)}{x^2} \to 0 \). Therefore:
\[
\frac{f(x)}{x} \to 0 \quad \text{as } x \to 0.
\]

\item 
\textbf{Conclusion:}\newline
Since \( \frac{f(h)}{h} \to 0 \) as \( h \to 0 \), it follows from the definition of the derivative that:
\[
f'(0) = 0.
\]

\end{enumerate}

\textbf{Intuition:}

The limit condition \(\lim_{x \to 0} \frac{f(x)}{x^2} = 0\) tells us that \( f(x) \) is much smaller than \( x^2 \) near 0. Because \( x \) grows much faster than \( f(x) \), the slope of \( f(x) \) at 0 (given by \( f'(0) \)) must also be 0.

\chapter{Definite integrals}

\section{Integral Region}

\subsection{A String of Inequality}

\(D={ (x,y) | 0 \leq x^2 \leq y \leq x \leq x  }\). How do you parse such an integral region? In other words, by splitting a string of formulas into easily recognizable formulas, I can draw the correct integration region based on the formulas.

To parse an integral region like \( D = { (x, y) \mid 0 \leq x^2 \leq y \leq x } \), we need to break down the inequalities into understandable parts and analyze their meaning step by step. Here’s how you can do it:

\hrulefill

\subsubsection{Step-by-Step Parsing}
\begin{enumerate}
\item 
\textbf{Expression Format}:
The region is described by:
\[
0 \leq x^2 \leq y \leq x.
\]

This means the constraints on \(x\) and \(y\) must satisfy all these inequalities simultaneously.

\end{enumerate}

\hrulefill
\begin{enumerate}
\item 
\textbf{Split the Inequalities}:
The compound inequality can be split into smaller parts:
\begin{itemize}
\item 
\(0 \leq x^2\): \(x^2\) is non-negative, which is always true for real \(x\).

\item 
\(x^2 \leq y\): \(y\) is greater than or equal to \(x^2\).

\item 
\(y \leq x\): \(y\) is less than or equal to \(x\).

\end{itemize}

\end{enumerate}

\hrulefill
\begin{enumerate}
\item 
\textbf{Understand the Range of \(x\)}:
\begin{itemize}
\item 
From \(y \leq x\), we know \(x\) acts as an upper limit for \(y\).

\item 
From \(x^2 \leq y\), \(y\) must also be greater than or equal to \(x^2\).

\item 
Since \(y \leq x\), \(x\) must be non-negative (\(x \geq 0\)), because \(x^2 \geq 0\) and \(y\) must lie between \(x^2\) and \(x\).

\end{itemize}

Therefore, the range of \(x\) is:
\[
x \in [0, 1],
\]
because \(y \leq x\) and \(y \geq x^2\) will collapse to \(x \in [0, 1]\) for the region to remain valid.

\end{enumerate}

\hrulefill
\begin{enumerate}
\item 
\textbf{Range of \(y\) for a Given \(x\)}:
For a fixed \(x\) in \([0, 1]\):
\begin{itemize}
\item 
\(y\) lies between \(x^2\) and \(x\):
\[
y \in [x^2, x].
\]

\end{itemize}

\end{enumerate}

\hrulefill
\begin{enumerate}
\item 
\textbf{Integration Region}:
The integration region \(D\) can now be described in rectangular coordinates:
\[
D = { (x, y) \mid 0 \leq x \leq 1, , x^2 \leq y \leq x }.
\]

\end{enumerate}

\hrulefill

\subsubsection{Visualization and Drawing}
\begin{enumerate}
\item 
\textbf{Boundaries}:
\begin{itemize}
\item 
The curve \(y = x^2\): Parabolic lower boundary.

\item 
The line \(y = x\): Linear upper boundary.

\end{itemize}

\item 
\textbf{Range}:
\begin{itemize}
\item 
\(x\) varies from \(0\) to \(1\), and \(y\) is sandwiched between \(x^2\) and \(x\).

\end{itemize}

\item 
\textbf{Shape}:
The region forms a thin strip between the parabola and the line in the first quadrant, bounded by \(x \in [0, 1]\).

\end{enumerate}

\subsubsection{Code Example (Matplotlib)}

If you want to plot the region:

\begin{minted}[frame=lines, fontsize=\normalsize]{python}
import numpy as np
import matplotlib.pyplot as plt

# Define the range of x
x = np.linspace(0, 1, 500)
y_lower = x**2  # y = x^2
y_upper = x     # y = x

# Plot the region
plt.fill_between(x, y_lower, y_upper, color='lightblue', label='Region D')

# Plot the curves
plt.plot(x, y_lower, label='y = x^2', color='blue')
plt.plot(x, y_upper, label='y = x', color='orange')

# Labels and legend
plt.xlabel('x')
plt.ylabel('y')
plt.legend()
plt.title('Integration Region D')
plt.grid(True)
plt.axis('equal')

plt.show()
\end{minted}

\section{Integral Symmetry Property}

\begin{bluebox}
Why \( \int_a^b f(x) dx = \int_a^b f(a+b-x) dx \)?
\end{bluebox}

The equality

\[
\int_a^b f(x) , dx = \int_a^b f(a+b-x) , dx
\]

is a result of a \textbf{change of variables} in the integral. Here's a step-by-step explanation:

\hrulefill

\textbf{1. Substitution}

Let \( u = a + b - x \). Then:
\begin{itemize}
\item 
As \( x \) ranges from \( a \) to \( b \), \( u \) ranges from \( a + b - a = b \) to \( a + b - b = a \).

\item 
The differential \( dx \) becomes \( du = -dx \).

\end{itemize}

\hrulefill

\textbf{2. Rewrite the Integral}

Using the substitution, we can rewrite the integral \( \int_a^b f(x) , dx \) as:
\[
\int_a^b f(x) , dx = \int_b^a f(a + b - u) (-du).
\]

\hrulefill

\textbf{3. Adjust the Limits}

Switching the limits of integration (from \( b \) to \( a \) into \( a \) to \( b \)) introduces a negative sign:
\[
\int_b^a f(a + b - u) (-du) = \int_a^b f(a + b - u) , du.
\]

\hrulefill

\textbf{4. Relabel \( u \) Back as \( x \)}

Since \( u \) is just a dummy variable of integration, we can relabel \( u \) as \( x \):
\[
\int_a^b f(a + b - u) , du = \int_a^b f(a + b - x) , dx.
\]

Thus, we have:
\[
\int_a^b f(x) , dx = \int_a^b f(a + b - x) , dx.
\]

\hrulefill

\textbf{Intuition}

This equality reflects a symmetry property: the substitution essentially "flips" the function \( f(x) \) about the midpoint \( \frac{a+b}{2} \) of the interval \([a, b]\).

\end{document}