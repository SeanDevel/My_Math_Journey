\documentclass[a4paper,12pt]{book}

\usepackage{amsmath}  % For math formulas
\usepackage{amssymb}  % For additional math symbols
\usepackage{amsfonts} % For more math fonts
\usepackage{geometry} % For adjusting page margins
\setcounter{secnumdepth}{2} % Allow numbering to depth 3 (subsubsection)
\geometry{margin=3cm} % 1 inch margins
\usepackage{hyperref} % For hyperlinks within the document
\usepackage{xcolor}
\usepackage{ulem}
\usepackage{tcolorbox}
\tcbuselibrary{skins}

\usepackage{minted}

\definecolor{light_blue}{RGB}{200, 240, 255} % Customize the shade as desired
\definecolor{light_gray}{RGB}{222, 222, 222}

% problem box
\newtcolorbox{pbox}{
	colback=light_blue,
	colframe=light_blue,
	boxrule=0pt,
	leftrule=2pt,
	rightrule=2pt,
	toprule=2pt,
	bottomrule=2pt,
	sharp corners,
}

\begin{document}
	
\title{Calculus Bites}
\author{Sean}
% \date{\today}

\maketitle

\tableofcontents

\newpage

\chapter{Integration By Parts}

\section{Example}

\begin{pbox}
\( \int (x+1) \cos{x} dx \).
\end{pbox}

To solve the indefinite integral \( \int (x+1) \cos{x} , dx \), we can use the method of integration by parts. The formula for integration by parts is:

\[
\int u , dv = uv - \int v , du
\]

We will choose \( u = x + 1 \) and \( dv = \cos{x} , dx \).
\begin{enumerate}
\item 
\textbf{Differentiate \( u \) and integrate \( dv \):}
\[
u = x + 1 \quad \Rightarrow \quad du = dx
\]
\[
dv = \cos{x} , dx \quad \Rightarrow \quad v = \int \cos{x} , dx = \sin{x}
\]

\item 
\textbf{Apply the integration by parts formula:}
\[
\int (x+1) \cos{x} , dx = (x+1) \sin{x} - \int \sin{x} , dx
\]

\item 
\textbf{Integrate \( \sin{x} \):}
\[
\int \sin{x} , dx = -\cos{x}
\]

\item 
\textbf{Substitute back:}
\[
\int (x+1) \cos{x} , dx = (x+1) \sin{x} + \cos{x} + C
\]

\end{enumerate}

Thus, the indefinite integral is:

\[
\int (x+1) \cos{x} , dx = (x+1) \sin{x} + \cos{x} + C
\]

\end{document}