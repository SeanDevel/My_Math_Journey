\documentclass[a4paper,12pt]{article}

\usepackage{amsmath}  % For math formulas
\usepackage{amssymb}  % For additional math symbols
\usepackage{amsfonts} % For more math fonts
\usepackage{geometry} % For adjusting page margins
\setcounter{secnumdepth}{2} % Allow numbering to depth 3 (subsubsection)
\geometry{margin=1in} % 1 inch margins
\usepackage{hyperref} % For hyperlinks within the document
\usepackage{xcolor}
\usepackage{ulem}
\usepackage{tcolorbox}
\tcbuselibrary{skins}

\definecolor{light_blue}{RGB}{200, 240, 255} % Customize the shade as desired
\definecolor{light_gray}{RGB}{222, 222, 222}

\newtcolorbox{qbox}{ % question box
	colback=light_blue,
	colframe=light_blue,
	boxrule=0pt,
	leftrule=2pt,
	rightrule=2pt,
	toprule=2pt,
	bottomrule=2pt,
	sharp corners,
}

\newtcolorbox{light_gray_box}{
colback=light_gray,
colframe=light_gray,
boxrule=0pt,
leftrule=2pt,
rightrule=2pt,
toprule=2pt,
bottomrule=2pt,
sharp corners,
}

\usepackage{enumitem}


\begin{document}
	
\title{Basis Calculus - Limit}
\author{Sean}

\maketitle

\tableofcontents

\newpage

\section{Oblique Asymptote}

\subsection{Derivation of Oblique Asymptote}

\begin{qbox}
For a univariate function \(f(x)\), the method to find its oblique asymptote is to first find the slope k, and then find the intercept b.

Where \( k=\lim_{x\to \infty} \frac{f(x)}{x} \);

\( b=\lim_{x\to \infty} (f(x)-kx) \).

Please explain the principle.
\end{qbox}

The method of finding the oblique asymptote for a univariate function \( f(x) \) involves determining the line \( y = kx + b \) that approximates the function as \( x \to \infty \). Here's the principle behind it:

\subsubsection{1. Definition of an Oblique Asymptote}

An oblique (or slant) asymptote is a straight line \( y = kx + b \) that the graph of the function \( y = f(x) \) approaches as \( x \to \infty \) (or \( x \to -\infty \)). For the line to be an asymptote, the vertical distance between the function and the line, \( |f(x) - (kx + b)| \), must approach 0 as \( x \to \infty \).

\subsubsection{2. Finding the Slope \( k \)}

The slope \( k \) represents the rate at which \( f(x) \) grows relative to \( x \). As \( x \to \infty \), the slope \( k \) is determined by the dominant term in \( f(x) \). Mathematically,

\[
k = \lim_{x \to \infty} \frac{f(x)}{x}.
\]

This ratio essentially captures the average rate of change of \( f(x) \) relative to \( x \). If \( k \) exists as \( x \to \infty \), it defines the slope of the oblique asymptote.
\begin{itemize}
\item 
If \( f(x) \) grows linearly with \( x \), \( k \) will be a finite value.

\item 
If \( f(x) \) grows faster than linearly (e.g., quadratically or exponentially), an oblique asymptote does not exist.

\end{itemize}

\subsubsection{3. Finding the Intercept \( b \)}

Once \( k \) is known, the intercept \( b \) is determined to complete the equation of the line \( y = kx + b \). The intercept \( b \) is found as the "remainder" after accounting for the linear growth represented by \( kx \). Specifically,

\[
b = \lim_{x \to \infty} \left(f(x) - kx\right).
\]

This calculation ensures that the vertical difference between \( f(x) \) and the line \( y = kx + b \) vanishes as \( x \to \infty \), thereby making \( y = kx + b \) the asymptote.

\subsubsection{4. Principle Behind the Method}

The principle hinges on decomposing the function \( f(x) \) into two parts:

\[
f(x) = kx + \text{remainder}(x),
\]

where the "remainder" is the term \( f(x) - kx \), which must approach a constant \( b \) as \( x \to \infty \). This decomposition isolates the dominant linear growth (\( kx \)) and ensures that any deviations from the linear trend shrink as \( x \) increases.
\begin{itemize}
\item 
The slope \( k \) captures the long-term trend of \( f(x) \) relative to \( x \).

\item 
The intercept \( b \) adjusts for the offset of \( f(x) \) from this linear trend.

\end{itemize}

In essence, this method provides a systematic way to identify the linear trend that \( f(x) \) follows as \( x \to \infty \).

\begin{qbox}
This ratio essentially captures the average rate of change of \( f(x) \) relative to \( x \). If \( k \) exists as \( x \to \infty \), it defines the slope of the oblique asymptote.

This narration is abstract and opaque.
\end{qbox}

Let me explain this in a simpler and clearer way with an example.

Imagine you have a function \( f(x) = 2x + 3 + \frac{1}{x} \). You want to find the oblique asymptote of this function, which is a line that the function \( f(x) \) gets very close to as \( x \to \infty \).

\hrulefill

\subsubsection{Step 1: Find the Slope \( k \)}

The slope \( k \) of the asymptote is given by:

\[
k = \lim_{x \to \infty} \frac{f(x)}{x}.
\]

Let’s compute this for \( f(x) = 2x + 3 + \frac{1}{x} \):

\[
\frac{f(x)}{x} = \frac{2x + 3 + \frac{1}{x}}{x} = 2 + \frac{3}{x} + \frac{1}{x^2}.
\]

As \( x \to \infty \), the terms \( \frac{3}{x} \) and \( \frac{1}{x^2} \) approach 0. So:

\[
k = \lim_{x \to \infty} \left(2 + \frac{3}{x} + \frac{1}{x^2}\right) = 2.
\]

The slope \( k = 2 \) means the asymptote is a line with slope 2.

\hrulefill

\subsubsection{Step 2: Find the Intercept \( b \)}

Now, we need the intercept \( b \). This is computed as:

\[
b = \lim_{x \to \infty} \left(f(x) - kx\right).
\]

Here, \( kx = 2x \). So:

\[
f(x) - kx = (2x + 3 + \frac{1}{x}) - 2x = 3 + \frac{1}{x}.
\]

As \( x \to \infty \), the term \( \frac{1}{x} \) approaches 0. Therefore:

\[
b = \lim_{x \to \infty} \left(3 + \frac{1}{x}\right) = 3.
\]

The intercept \( b = 3 \) means the asymptote crosses the \( y \)-axis at \( y = 3 \).

\hrulefill

\subsubsection{Final Result: The Oblique Asymptote}

The oblique asymptote is the line:

\[
y = kx + b = 2x + 3.
\]

\hrulefill

\subsubsection{Why This Method Works}
\begin{itemize}
\item 
The slope \( k = \lim_{x \to \infty} \frac{f(x)}{x} \) tells us how steeply \( f(x) \) grows relative to \( x \).

\item 
Subtracting \( kx \) removes the dominant linear growth, leaving just the "remainder" \( f(x) - kx \), which simplifies to a constant \( b \).

\end{itemize}

As \( x \to \infty \), \( f(x) \) becomes almost identical to the line \( y = 2x + 3 \), which is why this line is the asymptote.

\subsection{Two Statement}

\begin{qbox}
If a function has no horizontal asymptotes, then it has no oblique asymptotes.

If a function has no vertical asymptotes, then it has no oblique asymptotes.
\end{qbox}

\subsubsection{The First Statement}

\textbf{What this means}

\begin{itemize}
\item 
A horizontal asymptote exists if \( \lim_{x \to \infty} f(x) \) or \( \lim_{x \to -\infty} f(x) \) is a finite constant.

\item 
An oblique asymptote exists if \( f(x) \) can be approximated by a line \( y = kx + b \) as \( x \to \infty \) or \( x \to -\infty \).

\end{itemize}

\textbf{Analysis}

This statement is not true in general. A function can lack a horizontal asymptote but still have an oblique asymptote. Here’s why:
\begin{itemize}
\item 
Horizontal asymptotes occur when \( f(x) \) approaches a constant as \( x \to \infty \).

\item 
Oblique asymptotes occur when \( f(x) \) grows linearly with \( x \).

\end{itemize}

For example:
\begin{itemize}
\item 
\( f(x) = 2x + 3 + \frac{1}{x} \) has no horizontal asymptote because \( f(x) \to \infty \) as \( x \to \infty \).

\item 
However, \( f(x) \) has an oblique asymptote, \( y = 2x + 3 \), as shown in the earlier example.

\end{itemize}

Thus, the absence of a horizontal asymptote does not imply the absence of an oblique asymptote.

\hrulefill

\subsubsection{The Second Statement}

\textbf{What this means}

\begin{itemize}
\item 
A vertical asymptote exists when \( f(x) \to \pm \infty \) as \( x \to c \), where \( c \) is a specific finite value.

\item 
An oblique asymptote describes behavior as \( x \to \infty \) or \( x \to -\infty \).

\end{itemize}

\textbf{Analysis}

This statement is also not true. A function can lack vertical asymptotes but still have an oblique asymptote. Here’s why:
\begin{itemize}
\item 
Vertical asymptotes are related to the behavior of \( f(x) \) near certain finite \( x \)-values.

\item 
Oblique asymptotes describe the long-term (large \( |x| \)) behavior of \( f(x) \).

\end{itemize}

For example:
\begin{itemize}
\item 
\( f(x) = 2x + 3 + \frac{1}{x} \) has no vertical asymptotes because it is defined for all real \( x \).

\item 
However, \( f(x) \) has an oblique asymptote, \( y = 2x + 3 \).

\end{itemize}

Thus, the absence of vertical asymptotes does not imply the absence of oblique asymptotes.

\hrulefill

\subsubsection{Conclusion}

Both sentences are incorrect. Here's the corrected understanding:
\begin{enumerate}
\item 
Horizontal asymptotes and oblique asymptotes are independent. A function can lack one but have the other.

\item 
Vertical asymptotes are unrelated to the existence of oblique asymptotes. A function can lack vertical asymptotes and still have oblique asymptotes.

\end{enumerate}

\end{document}